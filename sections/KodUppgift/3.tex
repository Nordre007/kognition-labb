Beskriv också på vilket sätt man skulle kunna göra koden bättre för det särskilt
utvalda problemet

För att göra koden mer begriplig och läsbar bör man dela in de rader som hör ihop i block. Exempelvis bör metoder och attributer till klasser och traits ligga under själva namndeklarationen och indenterade ett steg till höger. Man kan även använda sig av krullparanteser för att göra det ännu tydligare var ett block av kod börjar och slutar. Om koden öppnas i en IDE som stöttar programmeringsspråket Scala kommer man även att få korrekt syntax-highlighting. Detta innebär att "orden" i koden kommer att få olika färger utifrån dess mening. Detta gör det mycket enklare att både medvetet och undermedvetet tolka koden. Dessutom blir kodens syfte mycket tydligare om man byter ut variabelnamnen till något som beskriver tydligare vad dess syfte är, eller bara något mer läsligt. Man bör även använda en begriplig och konsistent namngivningskonvention för ens kod. Exempelvis kan namn som består av ord skrivas så att de första bokstäverna hos alla ord förutom det första i namnet har stora bokstäver. Exempelvis blir timestamp till timeStamp och millisbetweenmoves till millisBetweenMoves. Namn till klasser, abstrakta klasser, interface och traits skrivs oftast med stor bokstav i början för att markera skilladen mellan dem och metoder och variabler. Till exempel blir entity till Entity och canmove till CanMove. Efter en undersökning där en grupp programmerare fick reflektera och testas till olika aspekter av läsligheten hos deras kod var de två vanligaste förändringarna i deras kod att de valde tydligare namn för variabler och metoder och att tänka mer på hur en annan person i framtiden kommer att läsa deras kod. \footnote{Todd Sedano, Code Readability Testing, an Empirical Study, 2016. p4}
